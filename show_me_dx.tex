% arara: xelatex
\documentclass[12pt]{article}

% \usepackage{physics}

\usepackage{hyperref}
\hypersetup{
    colorlinks=true,
    linkcolor=blue,
    filecolor=magenta,      
    urlcolor=cyan,
    pdftitle={Overleaf Example},
    pdfpagemode=FullScreen,
    }

\usepackage{verse}

\usepackage{tikzducks}

\usepackage{tikz} % картинки в tikz
\usepackage{tkz-euclide}
\usetikzlibrary{shapes, arrows, positioning}
\usepackage{microtype} % свешивание пунктуации

\usepackage{array} % для столбцов фиксированной ширины

\usepackage{indentfirst} % отступ в первом параграфе

\usepackage{sectsty} % для центрирования названий частей
\allsectionsfont{\centering}

\usepackage{amsmath, amsfonts, amssymb} % куча стандартных математических плюшек

\usepackage{comment}

\usepackage[top=2cm, left=1.2cm, right=1.2cm, bottom=2cm]{geometry} % размер текста на странице

\usepackage{lastpage} % чтобы узнать номер последней страницы

\usepackage{enumitem} % дополнительные плюшки для списков
%  например \begin{enumerate}[resume] позволяет продолжить нумерацию в новом списке
\usepackage{caption}

\usepackage{url} % to use \url{link to web}


\newcommand{\smallduck}{\begin{tikzpicture}[scale=0.3]
    \duck[
        cape=black,
        hat=black,
        mask=black
    ]
    \end{tikzpicture}}

\usepackage{fancyhdr} % весёлые колонтитулы
\pagestyle{fancy}
\lhead{}
\chead{Show me $dx$}
\rhead{}
\lfoot{}
\cfoot{}
\rfoot{}

\renewcommand{\headrulewidth}{0.4pt}
\renewcommand{\footrulewidth}{0.4pt}

\usepackage{tcolorbox} % рамочки!

\usepackage{todonotes} % для вставки в документ заметок о том, что осталось сделать
% \todo{Здесь надо коэффициенты исправить}
% \missingfigure{Здесь будет Последний день Помпеи}
% \listoftodos - печатает все поставленные \todo'шки


% более красивые таблицы
\usepackage{booktabs}
% заповеди из докупентации:
% 1. Не используйте вертикальные линни
% 2. Не используйте двойные линии
% 3. Единицы измерения - в шапку таблицы
% 4. Не сокращайте .1 вместо 0.1
% 5. Повторяющееся значение повторяйте, а не говорите "то же"


\setcounter{MaxMatrixCols}{20}
% by crazy default pmatrix supports only 10 cols :)


\usepackage{fontspec}
\usepackage{libertine}
\usepackage{polyglossia}

\setmainlanguage{english}
\setotherlanguages{russian}

% download "Linux Libertine" fonts:
% http://www.linuxlibertine.org/index.php?id=91&L=1
% \setmainfont{Linux Libertine O} % or Helvetica, Arial, Cambria
% why do we need \newfontfamily:
% http://tex.stackexchange.com/questions/91507/
% \newfontfamily{\cyrillicfonttt}{Linux Libertine O}

\AddEnumerateCounter{\asbuk}{\russian@alph}{щ} % для списков с русскими буквами
% \setlist[enumerate, 2]{label=\asbuk*),ref=\asbuk*}

%% эконометрические сокращения
\DeclareMathOperator{\Cov}{\mathbb{C}ov}
\DeclareMathOperator{\Corr}{\mathbb{C}orr}
\DeclareMathOperator{\Var}{\mathbb{V}ar}
\DeclareMathOperator{\col}{col}
\DeclareMathOperator{\row}{row}

\let\P\relax
\DeclareMathOperator{\P}{\mathbb{P}}

\DeclareMathOperator{\E}{\mathbb{E}}
% \DeclareMathOperator{\tr}{trace}
\DeclareMathOperator{\card}{card}
\DeclareMathOperator{\mgf}{mgf}

\DeclareMathOperator{\Convex}{Convex}
\DeclareMathOperator{\plim}{plim}

\usepackage{mathtools}
\DeclarePairedDelimiter{\norm}{\lVert}{\rVert}
\DeclarePairedDelimiter{\abs}{\lvert}{\rvert}
\DeclarePairedDelimiter{\scalp}{\langle}{\rangle}
\DeclarePairedDelimiter{\ceil}{\lceil}{\rceil}

\newcommand{\cN}{\mathcal{N}}
\newcommand{\cF}{\mathcal{F}}

\newcommand{\RR}{\mathbb{R}}
\newcommand{\NN}{\mathbb{N}}
\newcommand{\hb}{\hat{\beta}}
\newcommand{\dPois}{\mathrm{Pois}}


% arc, https://tex.stackexchange.com/questions/96680/
\makeatletter
\DeclareFontFamily{U}{tipa}{}
\DeclareFontShape{U}{tipa}{m}{n}{<->tipa10}{}
\newcommand{\arc@char}{{\usefont{U}{tipa}{m}{n}\symbol{62}}}%

\newcommand{\arc}[1]{\mathpalette\arc@arc{#1}}

\newcommand{\arc@arc}[2]{%
  \sbox0{$\m@th#1#2$}%
  \vbox{
    \hbox{\resizebox{\wd0}{\height}{\arc@char}}
    \nointerlineskip
    \box0
  }%
}
\makeatother




% dot of variable size, from
% https://tex.stackexchange.com/questions/389238/is-there-a-black-dot-symbol-that-i-can-use
\newcommand\vardot[1][.4]{\mathbin{\vcenter{\hbox{\scalebox{#1}{$\bullet$}}}}}
% dot above equality sign for Newton style
\newcommand{\ulteq}{\mathrel{\overset{\vardot}{=}}}
\newcommand{\ultsim}{\mathrel{\overset{\vardot}{\sim}}}
%\newcommand{\ulteq}{\mathrel{\overset{\infty}{=}}}
%\newcommand{\ultsim}{\mathrel{\overset{\infty}{\sim}}}


\newcommand{\angl}[1]{\widehat{#1}}
% \newcommand{\angl}[1]{\angle{#1}}

\begin{document}


\begin{verse}
    \begin{flushright}
fff
    \end{flushright}
\end{verse}


\begin{tcolorbox}[colback=yellow!50!red!25!white]
smth
\end{tcolorbox}

Для малых величин лучше использовать не $\Delta$, а $u$, $h$, $t$, \ldots{ }, чтобы избежать путаницы с треугольниками в геометрических пояснениях.

Обозначение для угла, $\angl{ABC}$.

\section*{На отрезке}

Десять лилипутов прыгают в длину независимо друг от друга на равномерное расстояние от $0$ до $1$,
обозначим их результаты прыжков как $X_1$, $X_2$, \dots, $X_{10}$.
Упорядочим эти результаты по возрастанию и получим $X_{(1)} \leq X_{(2)} \leq \dots \leq X_{(10)}$.

Найдите вероятность того, что результат бронзового призёра $X_{(3)}$ лежит в отрезке $[x; x+h]$ с точностью до $o(h)$.


\begin{tikzpicture}
  \tkzDefPoints{0/0/O, 6/0/X, 7/0/XH, 10/0/U}
  \tkzDrawSegment(O, U)
  %\tkzMarkSegment[pos=0, mark=|](O,U)
  %\tkzMarkSegment[pos=1, mark=x](O,U)
  \tkzDrawPoints(X, XH, O, U)
  \tkzLabelPoint[below](O){$0$\vphantom{$h$}}
  \tkzLabelPoint[below](X){$x$\vphantom{$h$}}
  \tkzLabelPoint[below](XH){$x+h$}
  \tkzLabelPoint[below](U){$1$\vphantom{$h$}}
\end{tikzpicture}

Для любой отдельной величины $X_i$ выполнено предельное равенство
\[
\P(X_i \in [x+h; 1]) \ulteq \P(X_i \in [x; 1]).
\]
Более того, если мы выберем произвольную точку $a$ на отрезке $[x; x+h]$, то в силу неравенства
\[
\P(X_i \in [x+h; 1]) \leq \P(X_i \in [a; 1]) \leq \P(X_i \in [x; 1])
\]
выполнено и предельное равенство
\[
\P(X_i \in [a; 1]) \ulteq \P(X_i \in [x; 1]).
\]
Вероятность $\P(X \in [x; 1])$ имеет более простую формулу, так как зависит только от $x$, поэтому мы будем использовать её.

Чтобы бронзовый призёр прыгнул на длину от $x$ до $x+h$ два прыгуна должны прыгнуть на меньшую длину, а семь оставшихся — на большую длину.
Мультиномиальный коэффициент $10!/(2! \, 1! \, 7!)$ показывает число способов разделить всех десятырех лилипутов на двоих с коротким прыжком, одного особого и семерых с длинным прыжком.
Следовательно,
\[
\P(X_{(3)} \in [x+h; 1]) \ulteq \frac{10!}{2! \, 1! \, 7!} \cdot \P(X_1, X_2 \leq [0; x]) \cdot \P(X_3 \in [x; x+h]) \cdot \P(X_4, X_5, \dots, X_{10} \in [x; 1]). 
\]
\[
\P(X_{(3)} \in [x+h; 1]) \ulteq \frac{10!}{2! \, 1! \, 7!} \cdot  h x^2 (1 - x)^7.
\]
Извлекаем функцию плотности из вероятности,
\[
f_{X_{(3)}}(x) = \begin{cases}
  \frac{10!}{2! \, 1! \, 7!} x^2 (1 - x)^7, \text{ если } x \in [0;1], \\
  0, \text{ иначе.} 
\end{cases}
\]



\section*{С окружностью}

Слепой Пью стоит в начале координат. 
Он стреляет один раз в случайном равномерно распределённом направлении от $0$ до $\pi$.

Стена проходит вдоль прямой $y = 1$.
Случайная величина $X$ — абсцисса точки попадания пули в стену. 

\begin{enumerate}
    \item Найдите вероятность $\P(X \in [x; x+ h])$ с точностью до $o(h)$.
    \item Найдите функцию плотности величины $X$. 
\end{enumerate}


Решение.


\begin{tikzpicture}

    % Angles A & B (can be modified)
    \def\AngleA{20} \def\AngleB{30}

    % Base points
    \tkzDefPoints{0/0/A,11/0/F,0/6/O,6.5/0/C}


    \tkzDefLine[orthogonal=through F](A,F) \tkzGetPoint{d}
    \tkzDefLine[orthogonal=through O](A,O) \tkzGetPoint{e}
    \tkzInterLL(O,e)(F,d) \tkzGetPoint{D}
    \tkzDrawSegment(A,D)

    \tkzDefLine[orthogonal=through C](A,F) \tkzGetPoint{b}
    \tkzInterLL(C,b)(O,D) \tkzGetPoint{B}
    \tkzDrawSegment(A,B)


    \tkzDrawSegment[dashed](A,O)
    \tkzDrawSegment[dashed](C,B)
    \tkzDrawSegment[dashed](F,D)

    \tkzDrawArc[rotate](A,O)(-90)

    \tkzInterLC(A,D)(A,O) \tkzGetSecondPoint{Ep}
    \tkzInterLC(A,B)(A,O) \tkzGetSecondPoint{Bo}
    
    \tkzDefLine[orthogonal=through Ep](A,D) \tkzGetPoint{bp}
    \tkzInterLL(Ep,bp)(A,B) \tkzGetPoint{Bp}

    \tkzDrawSegment(Ep,Bp)


    \tkzDefLine[orthogonal=through B](A,D) \tkzGetPoint{e}
    \tkzInterLL(B,e)(A,D) \tkzGetPoint{E}

    \tkzDrawSegment(E,B)

    \tkzDrawLine[add=0.1 and 0.1](A,F)
    \tkzDrawLine[add=0.1 and 0.1](O,D)

    \tkzLabelPoints[below](A,F,C,E)
    \tkzLabelPoint[below](Ep){$E_1$}
    \tkzLabelPoint[above](Bp){$B_1$}
    \tkzLabelPoint[left](Bo){$B_0$}
    \tkzLabelPoints[above](O,D,B)

    \tkzDrawPoints(A,F,O,D,C,B,Ep,Bp,Bo,E)

    \tkzMarkRightAngles[fill=blue!10,size=.25,draw](A,F,D B,E,D Bp,Ep,D)
    \tkzFillAngle[fill=orange!10,size=0.35,draw](F,A,D)
    \tkzFillAngle[fill=orange!10,size=0.35,draw](B,D,E)

    
\end{tikzpicture}

Найдём вероятность $\P(X \in [x; x + h])$.
Для этого мы от длины отрезка $BD$ перейдём к длине дуги $B_0 E_1$.

Треугольники $\triangle BED$ и $\triangle AFD$ подобны, поэтому 
\[
BE = BD \cdot \frac{DF}{AD} = h \frac{1}{\sqrt{1^2 + (x+h)^2}} \ulteq \frac{h}{\sqrt{1 + x^2}}.
\]
Треугольники $\triangle ABE$ и $\triangle AB_1E_1$ подобны, поэтому 
\[
B_0 E_1 \ulteq B_1E_1 = BE \cdot \frac{AB_1}{ AB} \ulteq  BE \cdot \frac{AB_0}{ AB} = \frac{BE}{\sqrt{1 + x^2}}
\]
Следовательно, $B_0E_1 \ulteq h / (1 + x^2)$, и искомая вероятность предельно равна
\[
\P(X \in [x; x + h]) = \frac{B_0E_1}{\pi} \ulteq \frac{h}{\pi (1 + x^2)}.
\]
Функция плотности равна 
\[
f(x) = \frac{1}{\pi(1 + x^2)}.
\]




Илон Маск гоняет на Cybertruck по окружности единичного радиуса с центром в начале координат.
В случайный момент времени момент времени он останавливается, точка его остановки равномерно распределена на окружности.
Обозначим абсциссу точки остановки с помощью $X$.
\begin{enumerate}
  \item Найдите вероятность $\P(X \in [x; x+ h])$ с точностью до $o(h)$ из геометрических соображений.
  \item Найдите функцию плотности и величины $X$.
\end{enumerate}

  \begin{tikzpicture}
    \tkzInit[xmax=5,ymax=5] % limits the size of the axes
    \tkzDrawX[>=latex]
    \tkzDrawY[>=latex]
    \tkzDefPoint(0,0){O}
    \tkzDefPoint(5,-0.5){Ab}
    \tkzDefPoint(3,0){X}
    \tkzDefPoint(0,3){Y}
    \tkzDefPoint(-0.5,5){Ac}
    \tkzDefPoint(2,3){Aprec}
    \tkzInterLC(O,Aprec)(O,Ab) \tkzGetPoints{antiA}{A}
    \tkzDefPoint(3,2){Bprec}
    \tkzInterLC(O,Bprec)(O,Ab) \tkzGetPoints{antiB}{B}

    \tkzDefPointBy[projection = onto O--X](A)\tkzGetPoint{D}
    \tkzDefPointBy[projection = onto O--X](B)\tkzGetPoint{E}
    \tkzDefPointBy[projection = onto O--Y](A)\tkzGetPoint{F}
    \tkzInterLL(A,F)(E,B) \tkzGetPoint{C}

    \tkzLabelPoints(O,A,B,D,E,C)
    \tkzDrawPoints(O,A,B,D,E,C)
    \tkzDrawSegment(A,B)
    \tkzDrawSegment(A,D)
    \tkzDrawSegment(B,E)
    \tkzDrawSegment(O,A)
    \tkzDrawSegment(O,B)
    \tkzDrawSegment(A,C)
    \tkzDrawSegment(C,B)


    \tkzDrawArc(O,Ab)(Ac)

  \end{tikzpicture}
  Искомая вероятность равна длине дуги $\arc{AB}$, делённой на $\pi$.

  Заметим, что $\angl{DAB} + \angl{BAC} = \pi/2$, $\angl{OAD} + \angl{DAB} \ulteq \pi/2$.
  Следовательно, $\angl{BAC} \ulteq \angl{OAD}$ и два треугольника предельно подобны, $\triangle ACB \ultsim \triangle ADO$.
  Из подобия следует, что $AB/AO \ulteq AC/AD$ и, следовательно,
  \[
  \P(X \in [x; x + h]) = \frac{\arc{AB}}{\pi} \ulteq \frac{AB}{\pi}  \ulteq \frac{AO \cdot AC/AD}{\pi} = \frac{h}{\pi\sqrt{1 - x^2}}.
  \] 
  Функция плотности равна
  \[
  f(x) = \begin{cases}
    \frac{1}{\pi\sqrt{1 - x^2}}, x \in [-1;1] \\
    0, \text{ иначе}.
  \end{cases}
  \]





\end{document}

